\chapter{Literaturverzeichnis}
\renewcommand{\bibname}{} % Entfernt den Bibliographietitel
\begingroup
\let\clearpage\relax % Deaktiviert Seitenwechsel
\bibliographystyle{plain}
\nopagebreak
\bibliography{refs}
\endgroup

\footnote{\cite{Chatterjee2015}, \cite{Song2004}, \cite{Quadros2011}, \cite{Singh2013}, \cite{Shakir2016}, \cite{Li2012}, \cite{Benhaddou2015}, \cite{Ferrer2023}, \cite{Ali2022}}


\chapter{Textzusammenfassung}\label{ch:intro}

Im Artikel \textit{A Survey of Secure Mobile Ad Hoc Routing Protocols} wird gezeigt, dass Sicherheitsaspekte 
bei Ad-hoc-Routing-Protokollen für mobile Netzwerke eine zentrale Rolle spielen und in welchen Situationen 
Ad-hoc-Routing sogar internetbasierten Messenger-Applikationen überlegen ist \cite[S. 1]{Li2007}.
Der Artikel beschreibt die gängigsten Ad-hoc-Routing-Protokolle, AODV, DSR, OLSR, TORA, ihre Einteilung in reaktiv (on-demand), proaktiv (table-driven)
und eine hybride Version beider, ihre grobe Verhaltensweise \cite[S. 2]{Li2007} sowie die Sicherheitsbedrohungen, 
denen sie durch Angriffe wie Black-Hole-, Wormhole- oder Sybil-Angriffe ausgesetzt sind, und welche Methoden geeignet sind, diese Schwachstellen zu 
beheben \cite[S. 4-11]{Li2007}.

Einige Methoden sind kryptografische Verfahren, Intrusion Detection Systems (IDS) und Trust-basierte Mechanismen und 
es wird darauf eingegangen, inwiefern sie die Sicherheit erhöhen können, ohne die Leistung 
des Netzwerks signifikant zu beeinträchtigen \cite[S. 7]{Li2007}. 

Das Besondere an diesem Textausschnitt ist die systematische Analyse bestehender Sicherheitslücken und die 
Diskussion potenzieller Lösungen für mobile Ad-hoc-Netzwerke \cite[S. 9]{Li2007}. Nach wie vor offen ist das 
Problem, unter welchen Bedingungen neue Sicherheitsmechanismen implementiert werden können, die sowohl skalierbar 
als auch effizient sind \cite[S. 12]{Li2007}. Folgende Fragen lässt Abusalah et al. jedoch offen: Wie können 
zukünftige Protokolle nicht nur sicher, sondern auch ressourcenschonend gestaltet werden, um den Einsatz in realen 
Anwendungen wie mobilen Messenger-Applikationen zu ermöglichen?

Darüber hinaus wird die Bedeutung von Vertrauen in der Sicherheit von MANETs hervorgehoben. 
Die Autoren betonen, dass Vertrauen eine wachsende Rolle spielt, insbesondere in offenen Umgebungen, 
in denen unbekannte Geräte jederzeit dem Netzwerk beitreten oder es verlassen können \cite[S. 13]{Li2007}. 
Schließlich wird darauf hingewiesen, dass bestehende Verschlüsselungsmechanismen oft zu ressourcenintensiv 
sind und daher nicht immer praktikable Lösungen darstellen \cite[S. 14]{Li2007}. Die Arbeit schließt mit der 
Empfehlung, zukünftige Forschungen auf die Entwicklung leichterer und effizienterer Sicherheitsmechanismen zu 
konzentrieren. 

Ein weiterer wichtiger Aspekt, den die Autoren hervorheben, ist die Notwendigkeit, Ad-hoc-Netzwerke an spezifische 
Anwendungsszenarien anzupassen, um eine optimale Balance zwischen Sicherheit und Leistung zu gewährleisten. 
Dabei wird auch auf die Bedeutung von Synergien zwischen Routing-Protokollen und Sicherheitsmechanismen eingegangen, 
um Bedrohungen proaktiv zu adressieren. Schließlich wird argumentiert, dass eine stärkere Integration von 
Lernmechanismen in Routing-Protokolle einen vielversprechenden Ansatz für die zukünftige Entwicklung darstellen könnte 
\cite[S. 15]{Li2007}.
