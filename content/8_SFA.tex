\chapter{Projekt SmartFactory@EH-Anlage}\label{ch:data}

\section{Projekt Smart Factory-Anlage: Aufgabenstellung}\label{sec:Projekt Smart Factory-Anlage: Aufgabenstellung}

Die SmartFactory-Anlage ist eine speziell für Schulungszwecke konzipierte
Automatisierungsanlage, die von der Abteilung Expert House entwickelt und betrieben wird. Sie beinhaltet verschiedene Produkte aus dem Siemens-Produktkatalog 
wie zum Beispiel speicherprogrammierbare Steuerungen, Lichtsensoren und Motoren.
Meine Aufgabe bestand darin, zusammen mit 18 weiteren Studierenden die
Weiterentwicklung für diese Anlage durchzuführen. Dies umfasste sowohl die
vollständige Ausbesserung und Behebung bereits bekannter Mängel, welche in einer List of Open Points (LOP) festgehalten wurden, 
als auch das Testen und Verbessern der Robustheit der Software. Die Software wurde nach dem 
ISA-88-Standard umgesetzt und alle Erweiterungen werden auch nach diesem hinzugefügt werden. Zur Programmierung wurde 
die Siemens-Automatisierungssoftware „TIA-Portal“ verwendet. Eine weitere Anforderung war es, die 
umfassenden Dokumentationsunterlagen in Form von Word-Dokumenten für nachfolgende Studierende auszubessern und 
gegebenenfalls mit den neuen Funktionen zu erweitern. Die Nutzung des ISA-88-Standard soll gewährleisten, das die Anlage 
ohne einen Ansprechpartner aus unserem Projektteam weiterentwickelbar ist. 



\section{Projekt Smart Factory-Anlage: Zielsetzung}\label{sec:Projekt Smart Factory-Anlage: Zielsetzung}

Das Ziel am Ende des Praxissemesters war es, die SmartFactory-Anlage weiterzuentwickeln, ihre Robustheit zu erhöhen und uns eine tiefes Verständnis für komplexe Systeme eigenständig zu erarbeiten,
um uns eine Grundlage für zukünftige Ausbildungs- und Weiterbildungsmodule zu schaffen. 
Dafür müssen für jedes Anlagenmodul bekannte Fehler und offene Punkte aus der List of Open Points (LOP) abgearbeitet werden, sowie bei der Erkennung neuer Fehler, 
diese zur List of Open Points (LOP) hinzugefügt werden. Des Weiteren müssen die Bedingungen wie eine einheitliche HMI-Schnittstelle 
und eine leichte Bedienung der Anlage umgesetzt werden.

\section{Projekt SmartFactory-Anlage: Projektorganisation}\label{sec:Projekt SmartFactory-Anlage: Projektorganisation (am ändern)}

Das Projektteam besteht aus 9 Informatikstudenten und 10 Elektro- und Informationstechnikstudenten, welche über die verschiedenden Anlagenteile und übergreifende Aufgaben verteilt wurden,
sowie drei Fachbetreuern. Diese haben uns als ``Product Owner'' im Projekt unterstützt und geholfen.  

Das gesamte Projekt wurde auf der agilen Methode \textbf{Scrum} basierend durchgeführt. Dabei wurden jedoch Anpassungen vorgenommen, die dem 
Projektablauf und den spezifischen Umständen gerecht wurden. Wie bereits in Abschnitt~\ref{sec:Einleitung}, „Einleitung“, erläutert, wurde die 
Arbeitsphase im Expert House sowohl bei uns Informatikern als auch bei den Elektro- und Informationstechnikern durch die \textbf{SPE-Phase} 
unterbrochen, allerdings zu unterschiedlichen Zeitpunkten. Dies führte zu einem geteilten Arbeitszeitplan: In den ersten acht Wochen arbeiteten 
alle gemeinsam am Projekt, in den darauffolgenden acht Wochen nur die Elektro- und Informationstechniker, und in den abschließenden acht Wochen 
waren ausschließlich wir Informatiker beteiligt.

Trotz der Abweichungen von den klassischen Scrum-Praktiken war dieser Ansatz sinnvoll. Durch die iterative Arbeitsweise und die klare Struktur 
von Scrum konnten wir den Überblick über die aktuellen Aufgaben besser behalten. Zur Organisation der Aufgaben und Projektmeilensteine half uns 
insbesondere ein \textbf{Kanban-Board}, das die Aufgaben nach den Kategorien strukturierte. Dieses visuelle 
Hilfsmittel erleichterte die Nachverfolgbarkeit von Fortschritten erheblich und förderte die Verständlichkeit während der regelmäßigen Meetings.

Die Kombination aus Scrum-Elementen und der Verwendung des Kanban-Boards sorgte für eine verbesserte Teamkommunikation und einen reibungsloseren 
Workflow. So konnte jeder Beteiligte stets nachvollziehen, welche Aufgaben anstanden, bearbeitet wurden oder bereits abgeschlossen waren. 
Besonders in der finalen Phase, als die Teams unabhängig voneinander arbeiteten, erwies sich diese Arbeitsweise als vorteilhaft, da sie die 
Selbstorganisation förderte und klare Prioritäten setzte.

Dies führte dazu, dass an den Übergabepunkten große und kleine Änderungen klar an das andere Projektteam übergeben werden mussten.
Gleichzeitig war es essenziell, die Dokumentation stets auf dem aktuellen Stand zu halten, um eine reibungslose Weiterarbeit 
zu ermöglichen.  

Dadurch stimmten wir uns gemeinsam, Betreuer und Auszubildende, in Daily-Standup-Meetings ab. Wir arbeiteten in Zwei-Wochen-Sprints, die jeweils mit einer Vorstellung der 
Ergebnisse an die Stakeholder endeten. Diese Präsentation ist ein wichtiges Mittel um unseren Fortschritt an der Anlage zu zeigen und Punkte für die nächste 
Bearbeitung festzulegen. Eine zentrale Rolle spielte die "List of Open Points" (LOP), die zur Dokumentation 
von bestehenden Fehlern und Mängeln, aber auch Funktionsanforderungen und Verbesserungen genutzt wird.

Die LOP war in Dringlichkeitsstufen unterteilt:
\begin{itemize}
    \item \textbf{Sehr wichtig:} Prozessbeendende Fehler oder solche, die mechanische Schäden verursachen könnten.
    \item \textbf{Mittel:} Fehler, die den Prozess nicht vollständig stoppen, aber zu Fehlern in der Verarbeitung von Flaschen führen.
    \item \textbf{Niedrig:} Fehler mit geringeren Auswirkungen, z. B. das Herausfallen einer Kugel bei der Initialisierung der Abfüllstation.
\end{itemize}

Die Stakeholder legten diese Prioritäten fest. Für jeden Sprint suchten sich die Verantwortlichen der jeweiligen Station 
die wichtigsten Aufgaben mit der höchsten Priorität heraus und arbeiteten diese ab.  

Gerade zu bearbeitende Aufgaben und bereits abgeschlossene wurden in ein Kanban-Board eingetragen. Dies half, den Fortschritt 
zu visualisieren und machte in den Daily-Meetings deutlich, welche Aufgaben eventuell vom Plan abwichen. Dadurch konnten 
notwendige Anpassungen schnell besprochen und vorgenommen werden, um weiterhin das Projektziel in der gegebenen Zeit zu 
erreichen.  

Das Kanban-Board wurde in folgende Bereiche unterteilt:
\begin{itemize}
    \item \textbf{Aufgaben:} Hier befinden sich alle Aufgaben, die im aktuellen Sprint erledigt werden müssten, mit denen 
    sich aber noch niemand befasst hat.
    \item \textbf{In Bearbeitung:} Hier wurden die aktuell bearbeiteten Aufgaben gesammelt.
    \item \textbf{Internes Review:} Wurde eine Aufgabe abgeschlossen, wurde das Ergebnis zunächst durch einen weiteren 
    Studierenden korrekturgelesen und auf Verständlichkeit überprüft.
    \item \textbf{Ready for Review:} Eine von einem Studierenden als in Ordnung befundene Aufgabe wird noch ein weiteres 
    Mal durch einen Betreuer geprüft.
    \item \textbf{Nacharbeiten:} Wurde eine durchgeführte Aufgabe entweder von einem Betreuer oder einem Studierenden als 
    nicht in Ordnung befunden, oder es traten im Projektverlauf Probleme damit auf, ist sie in diesem Bereich zu finden.
    \item \textbf{Erledigt:} In diesem Bereich befinden sich schließlich komplett abgeschlossene Aufgaben.
\end{itemize}

Des Weiteren haben wir mit dem Siemens-Multiuser-Server gearbeitet, eine angepasste Lösung für die Zusammenarbeit an 
Automatisierungsprojekten. Das bedeutet, dass jeder Studierende eine lokale 
Instanz des Projekts hatte, diese bearbeiten und an der Anlage testen konnte. Große Änderungen konnten dann auf einen 
GIT-ähnlichen Server hochgeladen werden, von dem alle anderen Instanzen diese Änderungen wieder herunterladen konnten. 
Außerdem konnte man über ein File-Locking-System bei Dateien, an denen man gerade arbeitete, festlegen, ob ein Konflikt 
besteht, wenn zwei Studierende gleichzeitig an derselben Datei arbeiten. Dies hat ``Merge-Konflikte'' und den Verlust 
von Code verhindert, dass die Zusammenarbeit erleichtert und ermöglicht eine grundlegende Versionskontrolle.

\section{Projekt SmartFactory-Anlage: Projektverlauf}
%------------------------------------------------------
\subsection{Auftaktwoche:}

Noch vor dem offiziellen Start des Projekts wurde in meiner Abteilung eine Auftaktwoche organisiert, die dazu diente, 
uns auf die bevorstehenden Aufgaben vorzubereiten. Der Schwerpunkt lag darauf, unser Wissen und unseren Umgang mit dem 
TIA Portal aufzufrischen. Obwohl wir als dual Studierende bereits einen zweiwöchigen Kurs zur Nutzung der Software 
absolviert hatten, fehlte uns durch ein Semester ohne praktische Anwendung die nötige Routine. In dieser Woche bestand 
unsere Aufgabe darin, die Steuerung eines Aufzugs mit fünf Stockwerken zu programmieren. Dabei wurden uns auch 
Programmierkonzepte wie die „State Machine“ vermittelt, die später eine wichtige Rolle bei der Anwendung an der 
SmartFactory-Anlage spielten.
%------------------------------------------------------
\subsection{Hands-On SmartFactory-Anlage:}

Nach der Auftaktwoche wurden wir zufällig in Teams von zwei bis drei Personen eingeteilt, um die SmartFactory-Anlage 
in Betrieb zu nehmen. Ziel war es, am Ende der Woche eine Flasche, mit dem bereits bestehenden Code, vollständig durch 
die Anlage zu führen. Jedem Team wurde dabei eine spezifische Station zugewiesen. Ich war für die Abfüllstation zuständig. 
Die Aufgabenstellung war bewusst vage gehalten, um uns die Möglichkeit zu geben, die Anlage eigenständig kennenzulernen. 
Das Ergebnis bei der Abfüllstation bestand darin, das angefragte Flaschen und Deckel richtig für erstellte Aufträge 
verwendet wurden und ein betriebsbeendender Fehler der Laufbänder behoben wurden. Allerdings wurden auch einige neue Punkte, 
wie die fehlerhafte Kalibrierung des Drehtellers oder die fehlende Initalisierungsfahrt, in die LOP (List of Open Points) 
nachgetragen. Nach Abschluss dieser Aufgabe führten wir ein „Lessons Learned“ durch. Die wichtigste Erkenntnis hierbei war, 
dass die Anlage ohne einen sorgfältig erarbeiteten Plan und ein durchdachtes Konzept nicht effektiv in Betrieb 
genommen werden kann.
%------------------------------------------------------
\subsection{Bibliotheksverantwortlicher (erweiternder Aufgabenbereich):} 

Mir wurde eine zusätzliche Aufgabe, in Form des Bibliotheksverantwortlichen, zugewiesen. Die Aufgabe bestand darin, 
die Projektbibliothek des Multiuser-Servers konsistent zu halten. Dies ist notwendig, da Programmbausteine gemäß dem 
ISA-88-Standard, wie beispielsweise \textit{Control Modules} oder \textit{Equipment Modules}, von mehreren Instanzen im 
Projekt genutzt werden können.

Ein Lichtsensor kann beispielsweise Teil eines Förderbands oder eines Drehtellers sein. Um doppelten Code zu vermeiden und 
die Erweiterbarkeit und Wiederverwendbarkeit so hoch wie möglich zu halten, wird im Code, wenn ein Lichtsensor benötigt wird, der Baustein aus der 
Projektbibliothek verwendet. Dieser Baustein wird im Projekt als neue Instanz des Bausteins aus der Projektbibliothek 
eingefügt. Möchte man den genannten Lichtsensor aktualisieren, werden alle Instanzen im Projekt sowie in der 
Projektbibliothek aktualisiert.

Da die Projektbibliothek versionsabhängig arbeitet, kann es vorkommen, dass jemand am Förderband eine Änderung vornimmt, die 
nicht mit der neuen Version des Lichtsensors kompatibel ist. Dies würde zu einem Ausfall der Lichtsensoren am Förderband 
führen. Ebenso könnte es passieren, dass auf dem Förderband noch eine ältere Version des Lichtsensors ohne die aktuellen 
Änderungen verwendet wird. Dies führt zu Inkonsistenzen, die von mir bereinigt werden mussten.
%------------------------------------------------------
\subsection{Flaschenmanagement in der Unit:}

\textbf{Problem:}  
Das Flaschenmanagement der Abfüllstation hatte zwei wesentliche Schwachstellen: Erstens wurden nicht immer ausreichend Flaschen geliefert, 
um einen Auftrag vollständig zu erfüllen. Beispielsweise wurden bei einem Auftrag über zwölf Flaschen nur neun bereitgestellt, wodurch die 
Abfüllung anhielt und auf die fehlenden Flaschen wartete (siehe Abbildung \ref{fig:Abfüllung}). Zweitens wurden die Auftragsdaten nicht remanent 
gespeichert, was bedeutete, dass nach jedem Neustart oder Kaltstart der Steuerung alle Flaschen aus dem Drehteller entfernt werden mussten, bevor 
ein neuer Auftrag gestartet werden konnte. Dies führte zu langen Stillstandszeiten und erschwerte die Bedienung der Anlage erheblich.  

Ein zusätzliches Problem ergab sich aus der Tatsache, dass der Drehteller nach einem Neustart erkannte, dass sich Flaschen in ihm befanden, 
jedoch keine gültigen Auftragsdaten vorlagen. Dadurch geriet der Drehteller in einen ungültigen Zustand und konnte die Aufgaben an den einzelnen 
Positionen nicht korrekt ausführen, was die gesamte Anlage blockierte.  

\textbf{Herausforderung:}  
Eine der zentralen Herausforderungen bestand darin, das Flaschenmanagement so anzupassen, dass sowohl die Lieferung unvollständiger 
Flaschenmengen als auch der Verlust von Auftragsdaten behoben werden konnte, ohne dabei die bestehenden Prozesse zu beeinträchtigen. 
Es war erforderlich, den Arbeitsfluss auch bei unvollständigen Flaschenlieferungen aufrechtzuerhalten und die Zustandsinformationen des 
Drehtellers nach einem Neustart korrekt zu initialisieren. Zudem musste eine Lösung gefunden werden, um unbekannte Flaschen im Drehteller 
zuverlässig zu erkennen und zu entfernen.  

\textbf{Lösung:}  
Um die Probleme bei der Flaschenlieferung zu beheben, wurde die Steuerung der Abfüllstation angepasst. Wenn innerhalb von sieben Sekunden 
keine neuen Flaschen geliefert werden, werden die bereits vorhandenen Flaschen im Drehteller dennoch abgefüllt und verschlossen. Dies gewährleistet einen kontinuierlichen Arbeitsfluss und minimiert Stillstandszeiten.  

Für das Problem der nicht remanenten Auftragsdaten wurde die Steuerung so erweitert, dass die Zustandsinformationen des Drehtellers nach 
einem Neustart korrekt initialisiert werden. Flaschen ohne gültige Auftragsdaten werden nun automatisch als unbekannt markiert und aus dem Drehteller entfernt. Diese Flaschen werden anschließend an das Quality Gate weitergeleitet, wodurch ein blockierter Zustand der Anlage verhindert wird.  

Die vorgenommenen Änderungen verbesserten die Effizienz der Abfüllstation erheblich und steigerten sowohl die Benutzerfreundlichkeit als 
auch die Robustheit der Anlage.

%------------------------------------------------------
\subsection{Kommunikation:} 

\textbf{Problem:}  
Die Kommunikation zwischen den einzelnen Stationen der Anlage war auf „On-Demand“-Basis ausgelegt. Das bedeutet, dass jede Station nur zu 
einem bestimmten Zeitpunkt mit einer anderen Station kommuniziert hat. Wenn jedoch eine Station nicht bereit war, die Kommunikation zu 
empfangen, ging diese „verloren“. Dies führte dazu, dass Anlagenteile nicht mehr synchron arbeiteten, die Anlage in einen fehlerhaften 
Zustand geriet und die Produktion gestoppt wurde.

\textbf{Herausforderung:}  
Hierfür musste ein neuer Baustein mit einem Watchdog-Timer eingeführt werden, der die Kommunikation zwischen den Stationen überwacht.

\textbf{Lösung:}  
Der neue Kommunikationsbaustein hat den alten Kommunikationsbaustein ersetzt, und die Logik wurde zu einer Acknowledge-Logik umgeschrieben. 
Bis ein Acknowledge von der anderen Station kommt, wird die Kommunikation nicht als abgeschlossen betrachtet. Dies führte dazu, dass die 
Produktion nach Wiederherstellung eines fehlerfreien Zustands der Anlage fortgesetzt werden kann und keine Kommunikation „verloren“ geht.
Dadurch kann auch die Flaschenproduktion nach einem Fehlerzustand automatisch von der Anlage fortgesetzt werden.

\textbf{Problem:}  
Es soll für andere Anlagenteile, vorallem aber der Kommissonierung möglich sein, Flaschen, welche von der Qualitätsprüfung am Quality-Gate als ungenügend beschreiben wurden, nachzubestellen. Diese Flaschen werden nämlich aussortiert und dadurch stehen nicht vollständige Auftrage in der Kommissonierung.

\textbf{Herausforderung:} 


\textbf{Lösung:}  
Es musste ein neuer Kommunikationsbaustein, für die Kommunikation mit der Kommisonierung, und eine neue Nachbestellfunktion implementiert werden. Der Kommunikationsbaustein war bereits vorhanden und es musste nur eine Kopie mit der IP der Kommisonierung in das Projekt eingefügt werden. Für die Funktion konnte die ungewandelte Funktion zum bestellen von Aufträgen benutzt und direkt in die Unit eingefügt werden, doch statt Eingaben vom HMI für die Anzahl der Flaschen und Kugeln zu erhalten, werden hier die Auftragsdaten der noch nachzubestellenden Flasche benutzt.
%------------------------------------------------------
\subsection{Drehteller:} 

\textbf{Problem:}  
Die Taster auf dem HMI-Bediener-Panal zur Steuerung des Drehtellers waren nicht korrekt eingestellt. Dies führte dazu, dass bei der Kalibrierung des Drehtellers ein 
Wechsel zwischen „drehe rechts“ und „drehe links“ nicht möglich war und zu einem nicht quittierbaren Fehler an der Abfüllstation führte. 
Beispielsweise drehte sich der Drehteller bei Betätigung des Knopfes „drehe rechts“ zwar in die gewünschte Richtung, setzte seine Bewegung 
jedoch auch nach dem Loslassen des Knopfes fort. Dies widersprach den Anforderungen an Sicherheit, Benutzerfreundlichkeit und Robustheit 
der Anlage.  

\textbf{Herausforderung:}  
Eine große Herausforderung bei dieser Aufgabe war es, die Fehlerursache zu identifizieren. Zunächst vermuteten wir, dass die Kommandos zum 
Anhalten des Drehtellers nicht korrekt gesendet wurden. Allerdings stellte sich heraus, dass die Kommandos nicht nur fehlerhaft gesendet wurden, 
sondern gar nicht implementiert waren. Stattdessen wurde eine Nebenwirkung des Drehtellers als Absolutwertgeber ausgenutzt, denn der Befehl, einen 
neuen Absolutwert an der aktuellen Position des Drehtellers zu setzen, wurde mit seiner höheren Priorität verwendet, um das Drehen nach links 
oder rechts zu stoppen.  

\textbf{Lösung:}  
Es musste ein neuer Befehl auf der untersten Steuerungsebene des Drehtellers implementiert werden, der das Stoppen des Drehtellers ermöglicht. 
Hierfür wurde im Control-Modul \texttt{CM\_PositioningDrive} das UDT (User Defined Type) um ein neues Kommando erweitert. Dies war erforderlich, 
damit im Equipment-Modul \texttt{EM\_Drehteller} ein entsprechender neuer Befehl hinzugefügt werden konnte.  

Als Technologieobjekt besitzt der Drehteller bereits vorgeschriebene Bibliotheksfunktionen. Durch das Einfügen dieser neuen Funktionalität konnte der 
Halte-Befehl korrekt in die Steuerung der Taster integriert werden. Mit der korrekten Implementierung des 
Halte-Befehls war es außerdem möglich, in der Anlagenlogik den zuvor missbräuchlich genutzten Befehl zum Setzen eines neuen Absolutwerts zu 
ersetzen. Dadurch konnte auch die Kalibrierung des Drehtellers ohne Benutzereingaben beim Kaltstart ermöglicht werden, da der remanent gespeicherte Absolutwert nicht beim Neustart überschrieben wurde und der Drehteller sich dadurch selbstständig korrigieren kann.  

\textbf{Problem:}
turntable beim anfahren und nothalt + Fehler des Technologieobjekts nicht quittierbar -> turntable verlässt ausgangsposition und wird neu kalibiriert (flaschen und auftragsposition stimmen nicht mehr über ein)/position des auftrages wird nicht übergeben, weil im error state keine rising edge des turntabels übergeben wird

Beim Anfahren der Drehscheibe sowie bei einem Not-Halt tritt ein Fehler im Technologieobjekt auf, der nicht quittiert werden kann. Dadurch verlässt die Drehscheibe ihre Ausgangsposition und muss neu kalibriert werden. Dies führt dazu, dass die Flaschen und die Auftragsposition nicht mehr übereinstimmen. Zudem wird die Position des Auftrags nicht übergeben, da im Fehlerzustand keine steigende Flanke des Drehtisches weitergegeben wird.

\textbf{Herausforderung:} 
beim nothalt spannung an der ep klemme wird abgeschaltet dadurch gibt es keine kontrolle mehr-> verbessrungswürdig) mc_power hat keinen Einfluss und damit auch mc_halt nicht

Während eines Not-Halts wird die Spannung an der EP-Klemme abgeschaltet, wodurch jegliche Kontrolle über die Drehscheibe verloren geht. Dies stellt eine Schwachstelle dar, da weder MC_Power noch MC_Halt Einfluss auf den Prozess haben.

\textbf{Lösung:}  
-> möglichkeiten: verzögerung der 24V am antrieb um halt zu ermöglichen, failsafe digitaleingänge auf der sinamics verdrahten (safty oder verdrahtung anpassen)
-> so wurde es gelöst: nachbestellen wenn kein deckel auf der flasche ist (baustein von nachbestellung mit statstaionsdata[8] und umformung als input) + flanke bei new order (flaschenanforderung) einführen
staterrorstatus ersetzt gegen temp (da stat über einen zyklus bestehen bleibt und den fehler nach quittieren wieder in den status schreibt)

Beim Anfahren der Drehscheibe sowie bei einem Not-Halt trat ein Fehler im Technologieobjekt auf, der nicht quittiert werden konnte. Infolgedessen verließ die Drehscheibe ihre Ausgangsposition und musste neu kalibriert werden, was dazu führte, dass die Flaschen und die Auftragsposition nicht mehr übereinstimmten. Zudem wurde die Position des Auftrags nicht korrekt übergeben, da im Fehlerzustand keine steigende Flanke des Drehtisches weitergegeben wurde.

Ein weiteres Problem bestand darin, dass während eines Not-Halts die Spannung an der EP-Klemme abgeschaltet wurde, wodurch jegliche Kontrolle über die Drehscheibe verloren ging. Dies stellte eine Schwachstelle dar, da weder MC_Power noch MC_Halt Einfluss auf den Prozess hatten.

Erarbeitete Lösungsansätze waren die Verzögerung der 24V-Versorgung am Antrieb, um ein kontrolliertes Anhalten zu ermöglichen, sowie die Anpassung der Verdrahtung der Failsafe-Digitaleingänge am Sinamics-System – entweder durch eine Safety-Schaltung oder durch eine direkte Verdrahtungsänderung. Aus Zeitgründen wurde sich jedoch gegen diese Ansätze entschieden.

Stattdessen wurde eine andere Lösung umgesetzt: Eine automatische Nachbestellung wird nun ausgelöst, wenn sich kein Deckel auf der Flasche befindet. Dies erfolgt durch einen speziellen Baustein für die Nachbestellung, der mit StatStationsData[8] arbeitet und die Werte entsprechend umformt. Zusätzlich wurde eine Flanke bei „New Order“ eingeführt, um die Flaschenanforderung korrekt zu erkennen. Darüber hinaus wurde der StatErrorStatus durch eine temporäre Variable (Temp) ersetzt, da der ursprüngliche Stat-Wert über einen Zyklus hinweg bestehen blieb und nach dem Quittieren den Fehler erneut in den Status schrieb.
%------------------------------------------------
\subsection{Kugelabfüllung:} 
\textbf{Problem:}
Rührer in den Kugelbehältern der Abfüllung hören nach zwei Minuten auf zu drehen, dadurch fallen keine Kugeln mehr zu den Vereinzlern um abgefüllt zu werden. Dadurch, da die Vereinzler nicht prüfen können ob eine Kugel sich in ihnen befindet, werden Flaschen nur halb oder garnicht abgefüllt.

\textbf{Herausforderung:}
Kurzschlussfehler nach l+ beheben.

\textbf{Lösung:}
Kurzschlussfehlertest ausschalten und 
Vereinzler automatische weiter arbeit bei einklemmen von kugel schnellere abfüllung war dadurch behindert -> logik anpassen und fehler abfangen (soll sich in die andere richtung bewegen und fehler aus der fehlerliste entfernen) (EM-Bottler)
%------------------------------------------------------
\subsection{Steuerung:}
\textbf{Problem:}

\textbf{Herausforderung:}

\textbf{Lösung:}
steuerung ein knopf in die anlage für turntable und gripper integriert (laufband und pneumatic war schon) -> button neu mappen und flankenauswertung zur steuerung einbauen
%------------------------------------------------------
\subsection{Kugelrückführung:}
\textbf{Problem:}
Neuer Pellet return neues Konzept, da sehr laut und umpräzise (wie davor wie jetzt) für Rückführung und neuer code

\textbf{Herausforderung:}
move_relative funktioniert nicht schrittmotor dreht nur 1,8 Grad und 12 löcher sind 30 Grad (30 nicht durch 1,8 teilbar) und generell ungenauer positionsgeber (dreht sich mehr oder weniger als 18 Grad)

\textbf{Lösung:}
->bin sensor im code nutzbar (inputmapping) in EM_PelletReturn
->velocity ist für movejog in der sw config anstatt eine static im cm positioning drive
->move_jog statt move_relative, wenn sensor aktiviert stope drehen und aktiviere farbsensor -> laufe weiter und 
%------------------------------------------------------
\section{Projekt SmartFactory-Anlage: Ergebnis}\label{sec:Projekt SmartFactory-Anlage: Ergebnis}

Der finale Funktionsumfang des Projekts lässt sich zum Zeitpunkt der Erstellung dieses Berichts noch nicht vollständig abschätzen, da einige 
Aspekte der Entwicklung und Implementierung noch in Arbeit sind. Dennoch konnte das Team bereits bedeutende Fortschritte erzielen und wesentliche 
Meilensteine gemäß den ursprünglichen Vorgaben und dem Projektplan erreichen. Besonders hervorzuheben ist, dass alle kritischen Fehler, die den 
Betrieb der Abfüllstation maßgeblich beeinträchtigen oder den Prozess unterbrechen konnten, erfolgreich identifiziert und behoben wurden. 
Lediglich kleinere Bugs, die keine direkte Auswirkung auf den kontinuierlichen Betrieb haben, verbleiben noch zur Bearbeitung. Damit wurde 
eines der zentralen Projektziele, die Robustheit der gesamten Anlage erheblich zu steigern, erfolgreich realisiert.

Der Fokus der weiteren Arbeiten liegt nun darauf, die Benutzerfreundlichkeit des Systems weiter zu verbessern und die Anlage so zu gestalten, 
dass sie zukünftig leicht erweitert und angepasst werden kann. Die Verbesserung der Bedienoberflächen, die Optimierung der Steuerungslogik und 
die Anpassung der Kommunikationsprozesse zwischen den Stationen sind dabei wichtige Arbeitsbereiche, die bereits angestoßen wurden.

Aus den regelmäßigen Teammeetings sowie dem wertvollen Feedback der Projektmitglieder und betreuenden Lehrkräfte geht klar hervor, dass auch an 
den anderen Stationen der Anlage signifikante Fortschritte erzielt wurden. Trotz der Herausforderungen, die sich durch die komplexen 
Abhängigkeiten der Systeme ergeben, konnte das Projektteam durch gezielte Koordination und klare Zielsetzungen einen stabilen Fortschritt 
sicherstellen. Sollte es dennoch vorkommen, dass der Projektfortschritt bis zum Abschluss des Praxissemesters nicht alle gesetzten Ziele 
vollständig erreicht, gewährleisten die sorgfältig erstellten Konzept- und Entwurfsdokumente eine solide Grundlage für die Weiterentwicklung 
des Projekts durch zukünftige Studierende. Diese Dokumente enthalten detaillierte Beschreibungen der bisherigen Arbeiten, der zugrunde liegenden 
Systemarchitektur sowie Empfehlungen für künftige Implementierungen.

Besonders bemerkenswert ist, dass durch die vorgenommenen Änderungen an der Abfüllstation nicht nur die Effizienz des Produktionsprozesses 
erheblich gesteigert werden konnte. Auch die Benutzerfreundlichkeit und Robustheit des gesamten Systems wurden signifikant verbessert. 
Die Reduktion von Störungen, die Erhöhung der Systemstabilität und die klar strukturierte Benutzerführung tragen dazu bei, dass die Anlage 
zuverlässiger und einfacher zu bedienen ist. Die ergriffenen Maßnahmen bilden somit eine wertvolle Grundlage für den langfristigen Erfolg 
des Projekts.
