\section{Persönliches Fazit zum Praktikum (umschreiben)}

Rückblickend möchte ich noch auf meine persönlichen Eindrücke des
Praxissemesters bei der Siemens AG eingehen.
Da ich dualer Student mit Berufserfahrung in einer anderen Firma bin, war für
mich die Arbeit in einem großen Industrieunternehmen keine gänzlich neue
Erfahrung. Allerdings unterscheidet sich die Arbeitsweise in meiner Fachabteilung,
dem Expert House deutlich davon, wie ich es bereits kannte. Besonders
hervorheben möchte ich die täglichen Meetings, durch welche es jederzeit möglich
war, einen ganzheitlichen Überblick über den Projektfortschritt zu haben. Ich
gelernt, dass verschiedene Projektmanagement-Modelle, wie z.B. das
Wasserfallmodell oder Kanban, nicht zwingend getrennt voneinander stattfinden
können, sondern das sinnvolle Methoden mehrerer dieser Modelle auch
kombiniert eingesetzt werden können. Auch Konzepte wie „Lessions learned“ und
„Expert Talks“ empfand ich als äußerst sinnvoll und hilfreich.
Etwas ungünstig war es aus meiner Sicht, dass die SPE-Phase mitten in der
Abteilungsphase stattfand und nicht zu Beginn oder am Ende. Dadurch ging aus
meiner Sicht etwas der Blick auf den Projektfortschritt der SmartFactory-Anlage
verloren. Neben den organisatorischen Aspekten empfand ich die
Programmierung mit dem TIA-Portal besonders interessant, da es sich dabei um
eine völlig andere Art der Programmierung im Vergleich zu den mir bekannten
Programmiersprachen handelt und viele Besonderheiten zu beachten sind, wie
zum Beispiel die Berücksichtigung des zyklischen Programmablaufs und dem
Bausteinkonzept. Mein persönliches Highlight war allerdings das Arbeiten an einer
Industrieanlage. Mir wurde klar, dass es ein besonderes Erfolgserlebnis ist, wenn
ein gut ausgearbeitetes Konzept, zu einem funktionsfähigen Code führt und
dadurch ein reibungsloser Prozessablauf an einer mit unzähligen Sensoren und
Aktoren ausgestatten Anlage möglich wird. Ist man neben der reinen Informatik
auch an Mechanik und Elektrotechnik interessiert wird, ist ein Praxissemester im
Expert House bei der Siemens AG sehr empfehlenswert.