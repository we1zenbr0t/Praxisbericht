\chapter{Persönliches Fazit zum Praktikum (umschreiben)}

Rückblickend möchte ich auf meine persönlichen Eindrücke des Praxissemesters bei der Siemens AG eingehen. Für mich war die Arbeit in einem großen 
Industrieunternehmen eine völlig neue und spannende Erfahrung. Besonders die Arbeitsweise in meiner Fachabteilung, dem Expert House, hat mich beeindruckt. 
Hervorheben möchte ich die täglichen Meetings, die es jederzeit ermöglichten, einen ganzheitlichen Überblick über den Projektfortschritt zu erhalten.

Ich habe gelernt, dass verschiedene Projektmanagement-Modelle, wie z. B. das Wasserfallmodell oder Kanban, nicht zwingend getrennt voneinander eingesetzt werden 
müssen, sondern dass eine sinnvolle Kombination der Methoden je nach Situation große Vorteile bringen kann. Auch Konzepte wie „Lessons learned“ und „Expert Talks“
empfand ich als äußerst sinnvoll und hilfreich.

Etwas ungünstig war es aus meiner Sicht, dass die SPE-Phase mitten in der Abteilungsphase stattfand und nicht zu Beginn oder am Ende. Dadurch ging meines 
Erachtens ein Teil des Blicks auf den Projektfortschritt der SmartFactory-Anlage verloren. Neben den organisatorischen Aspekten empfand ich die Programmierung mit
dem TIA-Portal als besonders interessant. Diese Art der Programmierung unterscheidet sich deutlich von klassischen Programmiersprachen und erfordert, 
beispielsweise durch den zyklischen Programmablauf und das Bausteinkonzept, besondere Aufmerksamkeit.

Mein persönliches Highlight war jedoch die Arbeit an einer Industrieanlage. Es ist ein besonderes Erfolgserlebnis, wenn ein gut durchdachtes Konzept in einen 
funktionierenden Code umgesetzt wird und dadurch ein reibungsloser Prozessablauf an einer Anlage mit unzähligen Sensoren und Aktoren ermöglicht wird. Wer neben
der reinen Informatik auch an Mechanik und Elektrotechnik interessiert ist, dem kann ich ein Praxissemester im Expert House bei der Siemens AG sehr empfehlen.
