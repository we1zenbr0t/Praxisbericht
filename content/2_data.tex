\chapter{Vorstellung der Siemens AG}\label{ch:data}

Die Siemens AG wurde 1847 von Werner Siemens und Georg Halske als „Telegraphen Bau-Anstalt von Siemens und Halske“ in Berlin 
gegründet.\footnote{Vgl. Ernst Klett Verlag: Infoblatt Siemens AG, 2004, \url{https://www.klett.de/alias/1036900}. Zugriff: 5. Januar 2025} 
Anfangs konzentrierte sich das Unternehmen auf die Entwicklung und den Vertrieb von Telegrafenapparaten, wodurch Siemens schnell zu einem 
bedeutenden Akteur in der Elektroindustrie wurde. In den darauffolgenden Jahren expandierte das Unternehmen nach Russland und 
England\footnote{Vgl. Merkur: Siemens – Geschichte, Aktie und Tätigkeitsfelder, 2022, \url{https://www.merkur.de/wirtschaft/siemens-geschichte-aktie-und-taetigkeitsfelder-91268844.html}. Zugriff: 19. Januar 2024} 
und erschloss neue Geschäftsfelder wie die Elektrifizierung der Eisenbahn.\footnote{Vgl. Ernst Klett Verlag, 2004}

In den 1930er Jahren erweiterte Siemens seine Produktpalette um elektrische Haushaltsgeräte, Wärme- und Heizungskonzepte sowie 
Medizintechnik.\footnote{Vgl. Merkur, 2022} Heute ist Siemens in vielen Branchen vertreten, darunter Industrie, Gebäudetechnik, Energietechnik, 
Mobilität, Gesundheitstechnik, Business Administration und Human Resource Management. Die Schwerpunkte des Unternehmens liegen auf der 
Digitalisierung, insbesondere der digitalen Transformation von Fertigungsprozessen, Transportnetzwerken, Gebäuden und Energiesystemen.

Die Siemens AG beschäftigt weltweit rund 320.000 Mitarbeiter und erzielte im Geschäftsjahr 2022/23 einen Umsatz von 74,8 Milliarden Euro sowie 
einen Nettogewinn von 8,5 Milliarden Euro.\footnote{Vgl. Siemens, 2023 zitiert nach de.statista.com, 
\url{https://de-statista-com.thn.idm.oclc.org/statistik/daten/studie/73827/umfrage/umsatz-von-siemens-seit-2005/}. Zugriff: 5. Januar 2025} 
Ein Jahr später, im Jahr 2024, lag der Umsatz bei 75,9 Milliarden Euro, was die anhaltende Stärke des Unternehmens unterstreicht.