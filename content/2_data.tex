\chapter{Vorstellung der Siemens AG (noch umschreiben)}\label{ch:data}

Siemens wurde 1847 von Werner Siemens und Georg Halske als „Telegraphen Bau-Anstalt von Siemens und Halske“ in Berlin gegründet.
\footnote{Vgl. Ernst Klett Verlag: Infoblatt Siemens AG, 2004, \url{https://www.klett.de/alias/1036900}. Zugriff: 5. Januar 2025} 
In den darauffolgenden Jahren expandierte die Firma in weitere Standorte in Russland und England \footnote{Vgl. Merkur: Siemens – Geschichte, 
Aktie und Tätigkeitsfelder, 2022, \url{https://www.merkur.de/wirtschaft/siemens-geschichte-aktie-und-taetigkeitsfelder-91268844.html}. 
Zugriff: 19. Januar 2024} und erschloss neue Geschäftsfelder, wie z.B. der Elektrifizierung der Eisenbahn. \footnote{Vgl. Ernst Klett Verlag, 2004}

In den 1930er Jahren kamen unter anderem Sparten wie elektrische Haushaltsgeräte, Wärme- und Heizungskonzepte sowie Medizintechnik hinzu.
\footnote{Vgl. Merkur, 2022} Heute sind die Bereiche Digital Industries, Smart Infrastructure und Healthineers die nach Anzahl der Mitarbeiter 
größten Segmente des Unternehmens. 

Die Siemens AG mit ihren aktuell 320.000 Mitarbeitern konnte im Geschäftsjahr 2022/23 einen Umsatz von rund 77,8 Millionen Euro erzielen und 
dabei einen Nettogewinn von 8,5 Millionen Euro erwirtschaften.\footnote{Vgl. Siemens, 2023 zitiert nach de.statista.com, 
\url{https://de-statista-com.thn.idm.oclc.org/statistik/daten/studie/73827/umfrage/umsatz-von-siemens-seit-2005/}. Zugriff: 5. Januar 2025}
